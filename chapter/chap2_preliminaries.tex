\clearpage
\setcounter{chapter}{1}
\chapter[{KIẾN THỨC NỀN TẢNG}]{KIẾN THỨC NỀN TẢNG}
\label{chapter:preliminaries}

Để giúp người đọc có cái nhìn rõ hơn về hệ thống, chúng tôi xin dành riêng một chương để trình bày về kiến thức nền tảng của hệ thống \textbf{InsightTune}.
Chương này sẽ trình bày về những kiến thức chúng tôi đã vận dụng trong quá trình làm việc, phân tích đặc tả yêu cầu, thiết lập kiến trúc hệ thống và xây
dựng hệ thống AI.

\clearpage

\input{chapter/chapter2/prerequisite.tex}
\input{chapter/chapter2/arch_knowledge.tex}
\input{chapter/chapter2/ai.tex}
\phantomsection
\chapter*{Kết chương}
Trong chương 2, chúng tôi đã liệt kê và giải thích đơn giản những khái niệm và kiến thức được vận dụng trong quá trình xây dựng hệ thống \textbf{InsightTune}.
Những kiến thức này sẽ được nhắc lại và sử dụng xuyên suốt các chương sau của báo cáo. Cụ thể, tại chương \ref{chapter:requirements}, 
kiến thức về thu thập phân tích đặc tả yêu cầu sẽ được
cụ thể hóa thông qua kết quả là yêu cầu, bảng biểu. Tiếp theo tại chương \ref{chapter:architecture}, những mô tả, phân tích về kiến 
trúc hệ thống sẽ được trình bày rõ ràng lấy nền tảng
là kiến thức trong chương \ref{chapter:preliminaries}. Do đó, việc hiểu và nắm chắc những kiến thức nền tảng tại chương này là bắt buộc để đọc hiểu phần tiếp theo của báo cáo này.