\section{Đặc tả ca sử dụng}

Trong phần này, những ca sử dụng được nhắc tới trong biểu đồ
ca sử dụng sẽ được mô tả chi tiết thông qua các đặc tả ca sử dụng
(use case specification).


\subsection{Phát bài hát}
Ca sử dụng phát bài hát là trọng tâm của một ứng dụng nghe 
nhạc. Chúng tôi đã phân tích và giải thích chi tiết trong 
đặc tả ca sử dụng dưới đây.


\begin{longtable}{|p{4.5cm}|p{10cm}|}
	\hline
	\textbf{Mã Use case} & UC-PlaySong \\
	\hline
	\textbf{Tên Use case} & Playing a song \\
	\hline
	\textbf{Tác nhân} & Người dùng \\
	\hline
	\textbf{Mô tả} & Người dùng bấm vào một bản nhạc trên màn hình chính để phát bản nhạc đó. \\
	\hline
	\textbf{Sự kiện kích hoạt} & Người dùng bấm vào bản nhạc \\
	\hline
	\textbf{Pre-condition} &
	\begin{itemize}
		\item Người dùng đã đăng nhập vào app và hình ảnh các bản nhạc đã được load trên màn hình chính.
		\item Kết nối internet ổn định.
	\end{itemize}
	\\
	\hline
	\textbf{Basic flow} &
	\begin{tabular}{|p{1cm}|p{7.5cm}|}
		\hline
		\textbf{STT} & \textbf{Sự kiện} \\
		\hline
		1 & Người dùng bấm vào hình bản nhạc trên màn hình chính \\
		\hline
		2 & Client (mobile app) gửi ID bản nhạc người dùng bấm vào về API Gateway của Server \\
		\hline
		3 & API Gateway gửi ID bản nhạc tới Track Service \\
		\hline
		4 & Track Service tìm trong cơ sở dữ liệu và trả về Storage Key của bản nhạc trên Amazon Cloud \\
		\hline
		5 & Client gọi Playing Service với Key để lấy link bản nhạc qua API Gateway \\
		\hline
		6 & Client load giao diện phát nhạc cho người dùng \\
		\hline
	\end{tabular}
	\\
	\hline
	\textbf{Alternate flow:} Client thoát giao diện phát nhạc &
	\begin{tabular}{|p{1cm}|p{7.5cm}|}
		\hline
		\textbf{STT} & \textbf{Sự kiện} \\
		\hline
		6.1.1 & Client thoát giao diện phát nhạc về home của ứng dụng \\
		\hline
		6.1.2 & Client hiện thanh điều khiển nhạc ở dưới cùng màn hình \\
		\hline
	\end{tabular}
	\\
	\hline
	\textbf{Post-condition} & Bản nhạc được phát và người dùng có thể điều khiển được bản nhạc \\
	\hline
	\textbf{Other} & Bản nhạc hiện trên màn hình chính được đảm bảo có trong database và được upload lên Object Storage Bucket \\
	\hline
	\caption{Use Case: Playing a song} \label{tab:usecase-playingsong} \\
\end{longtable}



\subsection{Đăng ký}
Ca sử dụng đăng ký cho phép người dùng tạo tài khoản để sử dụng ứng dụng.

\begin{longtable}{|p{4.5cm}|p{10cm}|}
	\hline
	\textbf{Mã Use case} & UC-Register \\
	\hline
	\textbf{Tên Use case} & Đăng ký\\
	\hline
	\textbf{Tác nhân} & Người dùng\\
	\hline
	\textbf{Mô tả} & Người dùng đăng ký để tạo tài khoản\\
	\hline
	\textbf{Sự kiện kích hoạt} & Người dùng bấm vào nút "Create An Account"\\
	\hline
	\textbf{Pre-condition} &
	\begin{itemize}
		\item Giao diện đăng ký được hiển thị.
		\item Kết nối internet ổn định.
	\end{itemize}
	\\
	\hline
	\textbf{Basic flow} &
	\begin{tabular}{|p{1cm}|p{7.5cm}|}
		\hline
		\textbf{STT} & \textbf{Sự kiện} \\
		\hline
		1 & Người dùng nhập email, password, confirmPassword, firstName, lastName vào form đăng ký \\
		\hline
		2 & Hệ thống kiểm tra định dạng và tính hợp lệ của các trường\\
		\hline
		3 & Nếu hợp lệ, hệ thống gửi thông tin đến server\\
		\hline
		4 & Server kiểm tra thông tin, nếu chưa tồn tại thì lưu thông tin và phản hồi kết quả \\
		\hline
		5 & Hệ thống thông báo đăng ký thành công và chuyển đến giao diện đăng nhập \\
		\hline
	\end{tabular}
	\\
	\hline
	\textbf{Alternate flow:} Các thao tác thay thế trong quá trình đăng ký &
	\begin{tabular}{|p{1cm}|p{7.5cm}|}
		\hline
		\textbf{STT} & \textbf{Sự kiện} \\
		\hline
		3.1 & Nếu thông tin không hợp lệ, hiển thị thông báo lỗi \\
		\hline
		4.1 & Nếu email đã tồn tại, hiển thị thông báo lỗi \\
		\hline
	\end{tabular}
	\\
	\hline
	\textbf{Post-condition} & Người dùng có thể đăng nhập bằng thông tin vừa đăng ký \\
	\hline
	\textbf{Other} &
    \begin{minipage}[t]{\linewidth}
	\begin{itemize}
		\item Email đăng ký được đảm bảo đúng định dạng và chưa từng được đăng ký trước đây.
		\item Thông tin yêu cầu được điền đầy đủ.
	\end{itemize}
    \end{minipage}
	\\
	\hline
	\caption{Use Case: Đăng ký} \label{tab:usecase-dangky} \\
\end{longtable}

\subsection{Đăng nhập}
Ca sử dụng đăng nhập cho phép người dùng đăng nhập vào ứng dụng để sử dụng.

\begin{longtable}{|p{4.5cm}|p{10cm}|}
	\hline
	\textbf{Mã Use case} & UC-Login \\
	\hline
	\textbf{Tên Use case} & Login \\
	\hline
	\textbf{Tác nhân} & Người dùng \\
	\hline
	\textbf{Mô tả} & Người dùng đăng nhập để sử dụng hệ thống. \\
	\hline
	\textbf{Sự kiện kích hoạt} & Người dùng bấm vào nút "Log in" \\
	\hline
	\textbf{Pre-condition} &
	\begin{itemize}
		\item Giao diện đăng nhập đã hiển thị.
		\item Kết nối internet ổn định.
	\end{itemize}
	\\
	\hline
	\textbf{Basic flow} &
	\begin{tabular}{|p{1cm}|p{7.5cm}|}
		\hline
		\textbf{STT} & \textbf{Sự kiện} \\
		\hline
		1 & Người dùng nhập email và mật khẩu vào form \\
		\hline
		2 & Người dùng nhấn nút "Log in" \\
		\hline
		3 & Hệ thống gửi thông tin đến server \\
		\hline
		4 & Server kiểm tra thông tin và phản hồi kết quả \\
		\hline
		5 & Client load giao diện home của ứng dụng cho người dùng sử dụng \\
		\hline
	\end{tabular}
	\\
	\hline
	\textbf{Alternate flow:} Đăng nhập cách khác & 2.1 Người dùng nhấn nút "Login with Google" để đăng nhập bằng Google
	\\
	\hline
	\textbf{Post-condition} & 
	\begin{itemize}
		\item Hệ thống khởi tạo phiên làm việc hợp lệ cho người dùng.
		\item Giao diện chính được hiển thị để bắt đầu sử dụng hệ thống.
	\end{itemize} \\
	\hline
	\textbf{Other} & Thông tin đăng nhập hợp lệ \\
	\hline
	\caption{Use Case: Login} \label{tab:usecase-login} \\
\end{longtable}

\subsection{Đăng xuất}
Ca sử dụng đăng xuất cho phép người dùng đăng xuất tài khoản ra khỏi ứng dụng.
\begin{longtable}{|p{4.5cm}|p{10cm}|}
	\hline
	\textbf{Mã Use case} & UC-Logout \\
	\hline
	\textbf{Tên Use case} & Logout \\
	\hline
	\textbf{Tác nhân} & Người dùng \\
	\hline
	\textbf{Mô tả} & Người dùng đăng xuất khỏi hệ thống. \\
	\hline
	\textbf{Sự kiện kích hoạt} & Người dùng bấm vào nút "Log out" \\
	\hline
	\textbf{Pre-condition} &
	\begin{itemize}
		\item Người dùng đã đăng nhập thành công.
		\item Kết nối internet ổn định.
	\end{itemize}
	\\
	\hline
	\textbf{Basic flow} &
	\begin{tabular}{|p{1cm}|p{7.5cm}|}
		\hline
		\textbf{STT} & \textbf{Sự kiện} \\
		\hline
		1 & Người dùng ấn vào "Setting" \\
		\hline
		2 & Người dùng nhấn nút "Log out" \\
		\hline
		3 & Hệ thống gửi thông tin đến server \\
		\hline
		4 & Server kiểm tra thông tin và xóa token \\
		\hline
		5 & Client load giao diện bắt đầu \\
		\hline
	\end{tabular}
	\\
	\hline
	\textbf{Alternate flow:} & None
	\\
	\hline
	\textbf{Post-condition} & 
	\begin{itemize}
		\item Giao diện bắt đầu được hiển thị.
	\end{itemize} \\
	\hline
	\textbf{Other} & None \\
	\hline
	\caption{Use Case: Logout} \label{tab:usecase-logout} \\
\end{longtable}


\subsection{Tìm kiếm bài hát}
Ca sử dụng tìm kiếm bài hát cho phép người dùng có thể tìm kiếm các bài hát theo tên bài hát hoặc tên tác giả.
\begin{longtable}{|p{4.5cm}|p{10cm}|}
	\hline
	\textbf{Mã Use case} & UC-SearchSong \\
	\hline
	\textbf{Tên Use case} & Seach songs \\
	\hline
	\textbf{Tác nhân} & Người dùng \\
	\hline
	\textbf{Mô tả} & Người dùng tìm kiếm bài hát. \\
	\hline
	\textbf{Sự kiện kích hoạt} & Người dùng nhập keyword vào ô search \\
	\hline
	\textbf{Pre-condition} &
	\begin{itemize}
		\item Người dùng đã đăng nhập thành công.
		\item Kết nối internet ổn định.
	\end{itemize}
	\\
	\hline
	\textbf{Basic flow} &
	\begin{tabular}{|p{1cm}|p{7.5cm}|}
		\hline
		\textbf{STT} & \textbf{Sự kiện} \\
		\hline
		1 & Người dùng ấn vào nút "Search" \\
		\hline
		2 & Người dùng nhập tên bài hát \\
		\hline
		3 & Hệ thống gửi tên bài hát đến server \\
		\hline
		4 & Server kiểm tra thông tin và trả về kết quả \\
		\hline
		5 & Client load các bài hát lên giao diện \\
		\hline
	\end{tabular}
	\\
	\hline
	\textbf{Alternate flow:} Tìm kiếm theo tác giả & 2.1 Người dùng nhập tên tác giả \\
	\hline
	\textbf{Post-condition} & 
	\begin{itemize}
		\item Các bài hát được hiển thị trên giao diện và người dùng có thể phát nhạc.
	\end{itemize} \\
	\hline
	\textbf{Other} & None \\
	\hline
	\caption{Use Case: Searching Songs} \label{tab:usecase-searchsong} \\
\end{longtable}

\subsection{Xem bài hát gần đây}
Ca sử dụng xem bài hát gần đây cho phép người dùng có thể xem lịch sử phát bài hát.
\begin{longtable}{|p{4.5cm}|p{10cm}|}
	\hline
	\textbf{Mã Use case} & UC-HistorySong \\
	\hline
	\textbf{Tên Use case} & History songs \\
	\hline
	\textbf{Tác nhân} & Người dùng \\
	\hline
	\textbf{Mô tả} & Người dùng xem lịch sử bài hát. \\
	\hline
	\textbf{Sự kiện kích hoạt} & Người dùng vào giao diện home của ứng dụng \\
	\hline
	\textbf{Pre-condition} &
	\begin{itemize}
		\item Người dùng đã đăng nhập thành công.
		\item Kết nối internet ổn định.
	\end{itemize}
	\\
	\hline
	\textbf{Basic flow} &
	\begin{tabular}{|p{1cm}|p{7.5cm}|}
		\hline
		\textbf{STT} & \textbf{Sự kiện} \\
		\hline
		1 & Người dùng vào giao diện home của ứng dụng \\
		\hline
		2 & Client gửi thông tin người dùng lên server để lấy lịch sử nghe \\
		\hline
		3 & Server kiểm tra thông tin và trả về kết quả \\
		\hline
		4 & Client load các bài hát server trả về lên giao diện \\
		\hline
	\end{tabular}
	\\
	\hline
	\textbf{Alternate flow:} & None \\
	\hline
	\textbf{Post-condition} & 
	\begin{itemize}
		\item Các bài hát được hiển thị trên giao diện và người dùng có thể phát nhạc.
	\end{itemize} \\
	\hline
	\textbf{Other} & Đảm bảo người dùng đã từng phát nhạc trước đó \\
	\hline
	\caption{Use Case: History Songs} \label{tab:usecase-historysong} \\
\end{longtable}

\subsection{Lưu bài hát yêu thích}
Ca sử dụng lưu bài hát yêu thích cho phép người dùng có thể lưu các bài hát vào mục yêu thích.
\begin{longtable}{|p{4.5cm}|p{10cm}|}
	\hline
	\textbf{Mã Use case} & UC-FavoriteSong \\
	\hline
	\textbf{Tên Use case} & Favorite songs \\
	\hline
	\textbf{Tác nhân} & Người dùng \\
	\hline
	\textbf{Mô tả} & Người dùng lưu bài hát yêu thích. \\
	\hline
	\textbf{Sự kiện kích hoạt} & Người dùng ấn nút "Favorite" \\
	\hline
	\textbf{Pre-condition} &
	\begin{itemize}
		\item Người dùng đã đăng nhập thành công.
		\item Người dùng đang phát nhạc và giao diện bài hát đã hiển thị.
		\item Kết nối internet ổn định.
	\end{itemize}
	\\
	\hline
	\textbf{Basic flow} &
	\begin{tabular}{|p{1cm}|p{7.5cm}|}
		\hline
		\textbf{STT} & \textbf{Sự kiện} \\
		\hline
		1 & Người dùng nhấn nút "Favorite" ở góc dưới cùng bên phải màn hình \\
		\hline
		2 & Client gửi thông tin bài hát, người dùng lên server để thêm vào mục yêu thích \\
		\hline
		3 & Server kiểm tra thông tin và lưu vào CSDL \\
		\hline
		4 & Server trả về kết quả \\
		\hline
	\end{tabular}
	\\
	\hline
	\textbf{Alternate flow:} & None \\
	\hline
	\textbf{Post-condition} & 
	\begin{itemize}
		\item Bài hát được thêm vào mục yêu thích.
	\end{itemize} \\
	\hline
	\textbf{Other} & None \\
	\hline
	\caption{Use Case: Favorite Songs} \label{tab:usecase-favoritesong} \\
\end{longtable}

\subsection{Sửa thông tin cá nhân}
Ca sử dụng sửa thông tin cá nhân cho phép người dùng có thể sửa các thông tin firstname, lastname, phone, address và mật khẩu.
\begin{longtable}{|p{4.5cm}|p{10cm}|}
	\hline
	\textbf{Mã Use case} & UC-UpdateProfile \\
	\hline
	\textbf{Tên Use case} & Update Profile \\
	\hline
	\textbf{Tác nhân} & Người dùng \\
	\hline
	\textbf{Mô tả} & Người dùng có thể sửa các thông tin firstname, lastname, address, phone và mật khẩu tài khoản. \\
	\hline
	\textbf{Sự kiện kích hoạt} & Người dùng vào trang cài đặt, ấn nút "Save" hoặc "Change Password" nếu đổi mật khẩu \\
	\hline
	\textbf{Pre-condition} &
	\begin{itemize}
		\item Người dùng đã đăng nhập thành công.
		\item Kết nối internet ổn định.
	\end{itemize}
	\\
	\hline
	\textbf{Basic flow} &
	\begin{tabular}{|p{1cm}|p{7.5cm}|}
		\hline
		\textbf{STT} & \textbf{Sự kiện} \\
		\hline
		1 & Người dùng nhấn vào avatar ở góc trên cùng bên phải màn hình \\
		\hline
		2 & Người dùng chọn "Edit Profiles" \\
		\hline
		3 & Người dùng nhập thông tin vào form đổi thông tin \\
		\hline
		4 & Server lưu kết quả \\
		\hline
		5 & Giao diện hiển thị thông tin đã cập nhật \\
		\hline
	\end{tabular}
	\\
	\hline
	\textbf{Alternate flow:} Người dùng thay đổi mật khẩu &
	\begin{tabular}{|p{1cm}|p{7.5cm}|}
		\hline
		\textbf{STT} & \textbf{Sự kiện} \\
		\hline
		2.1.1 & Người dùng chọn "Change Password" \\
		\hline
		3.1.2 & Người dùng nhập thông tin vào form đổi mật khẩu \\
		\hline
		2.2.1 & Người dùng nhấn vào avatar \\
		\hline
		3.2.2 & Người dùng chọn avatar mới \\
		\hline
	\end{tabular}
	\\
	\hline
	\textbf{Post-condition} & Thông tin người dùng được cập nhật \\
	\hline
	\textbf{Other} & Giao diện hiện thông tin đã cập nhật \\
	\hline
	\caption{Use Case: Update Personal Information} \label{tab:usecase-updateprofile} \\
\end{longtable}


