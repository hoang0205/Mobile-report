\section{Đặc tả ca sử dụng}

Trong phần này, những ca sử dụng được nhắc tới trong biểu đồ
ca sử dụng sẽ được mô tả chi tiết thông qua các đặc tả ca sử dụng
(use case specification).


\subsection{Phát bài hát}
Ca sử dụng phát bài hát là trọng tâm của một ứng dụng nghe 
nhạc. Chúng tôi đã phân tích và giải thích chi tiết trong 
đặc tả ca sử dụng dưới đây.


\begin{longtable}{|p{4.5cm}|p{10cm}|}
	\hline
	\textbf{Mã Use case} & UC-PlaySong \\
	\hline
	\textbf{Tên Use case} & Playing a song \\
	\hline
	\textbf{Tác nhân} & Người dùng \\
	\hline
	\textbf{Mô tả} & Người dùng bấm vào một bản nhạc trên màn hình chính để phát bản nhạc đó. \\
	\hline
	\textbf{Sự kiện kích hoạt} & Người dùng bấm vào bản nhạc \\
	\hline
	\textbf{Pre-condition} &
	\begin{itemize}
		\item Người dùng đã đăng nhập vào app và hình ảnh các bản nhạc đã được load trên màn hình chính.
		\item Kết nối internet ổn định.
	\end{itemize}
	\\
	\hline
	\textbf{Basic flow} &
	\begin{tabular}{|p{1cm}|p{7.5cm}|}
		\hline
		\textbf{STT} & \textbf{Sự kiện} \\
		\hline
		1 & Người dùng bấm vào hình bản nhạc trên màn hình chính \\
		\hline
		2 & Client (mobile app) gửi ID bản nhạc người dùng bấm vào về API Gateway của Server \\
		\hline
		3 & API Gateway gửi ID bản nhạc tới Track Service \\
		\hline
		4 & Track Service tìm trong cơ sở dữ liệu và trả về Storage Key của bản nhạc trên Amazon Cloud \\
		\hline
		5 & Client gọi Playing Service với Key để lấy link bản nhạc qua API Gateway \\
		\hline
		6 & Client load giao diện phát nhạc cho người dùng \\
		\hline
	\end{tabular}
	\\
	\hline
	\textbf{Alternate flow:} Client thoát giao diện phát nhạc &
	\begin{tabular}{|p{1cm}|p{7.5cm}|}
		\hline
		\textbf{STT} & \textbf{Sự kiện} \\
		\hline
		6.1.1 & Client thoát giao diện phát nhạc về home của ứng dụng \\
		\hline
		6.1.2 & Khi người dùng vào lại, Client hiện thanh điều khiển nhạc ở dưới cùng màn hình \\
		\hline
	\end{tabular}
	\\
	\hline
	\textbf{Post-condition} & Bản nhạc được phát và người dùng có thể điều khiển được bản nhạc \\
	\hline
	\textbf{Other} & Bản nhạc hiện trên màn hình chính được đảm bảo có trong database và được upload lên Object Storage Bucket \\
	\hline
	\caption{Use Case: Playing a song} \label{tab:usecase-playingsong} \\
\end{longtable}



\subsection{Sửa danh mục}
\begin{longtable}{|p{4.5cm}|p{10cm}|}
	\hline
	\textbf{Mã Use case} & 9 \\
	\hline
	\textbf{Tên Use case} & Sửa danh mục\\
	\hline
	\textbf{Tác nhân} & Người dùng \\
	\hline
	\textbf{Mô tả} & Người dùng sửa một danh mục đã tạo\\
	\hline
	\textbf{Điều kiện kích hoạt} & Bấm icon sửa bên cạnh một danh mục trong phần "All category"\\
	\hline
	\textbf{Pre-condition} &  Người dùng đã đăng nhập\\
	\hline
	\textbf{Post-condition} &  Không\\
	\hline
	\textbf{Basic flow} &
	\begin{tabular}{|p{1cm}|p{7.5cm}|}
		\hline
		\textbf{STT} & \textbf{Sự kiện} \\
		\hline
		1 & Người dùng chọn danh mục cần sửa \\
		\hline
		2 & Hệ thống hiển thị thông tin danh mục\\
		\hline
		3 & Người dùng chỉnh sửa thông tin danh mục và xác nhận\\
		\hline
		4 & Hệ thống cập nhật danh mục và lưu vào CSDL \\
		\hline
	\end{tabular}
	\\
	\hline
	\textbf{Alternate flow:} Người dùng nhập thông tin lỗi &
	\begin{tabular}{|p{1cm}|p{7.5cm}|}
		\hline
		\textbf{STT} & \textbf{Sự kiện} \\
		\hline
		4b & Hệ thống yêu cầu nhập lại thông tin\\
		\hline
	\end{tabular}
	\\
	\hline
	\caption{Use Case: Sửa danh mục} \label{tab:usecase-xoaloaigiaodich} \\
\end{longtable}

\subsection{Báo cáo chi tiêu}
\begin{longtable}{|p{4.5cm}|p{10cm}|}
	\hline
	\textbf{Mã Use case} & 10 \\
	\hline
	\textbf{Tên Use case} & Báo cáo chi tiêu\\
	\hline
	\textbf{Tác nhân} & Người dùng \\
	\hline
	\textbf{Mô tả} & Báo cáo thu chi của người dùng trong một khoảng thời gian \\
	\hline
	\textbf{Điều kiện kích hoạt} & Bấm nút "Statistics" \\
	\hline
	\textbf{Pre-condition} & Người dùng đã đăng nhập\\
	\hline
	\textbf{Post-condition} & Người dùng nhận được báo cáo\\
	\hline
	\textbf{Basic flow} &
	\begin{tabular}{|p{1cm}|p{7.5cm}|}
		\hline
		\textbf{STT} & \textbf{Sự kiện} \\
		\hline
		1 & Người dùng vào phần "Statistics" \\
		\hline
		2 & Người dùng chọn khoảng thời gian muốn thống kê \\
		\hline
		3 & Hệ thống truy vấn và hiển thị biểu đồ giá trị thu chi theo danh mục\\
		\hline
	\end{tabular}
	\\
	\hline
	\caption{Use Case: Báo cáo chi tiêu} \label{tab:usecase-baocaochitieu} \\
\end{longtable}

\subsection{Trò chuyện với chatbot}
\begin{longtable}{|p{4.5cm}|p{10cm}|}
	\hline
	\textbf{Mã Use case} & 11 \\
	\hline
	\textbf{Tên Use case} & Trò chuyện với chatbot \\
	\hline
	\textbf{Tác nhân} & Người dùng, Chatbot \\
	\hline
	\textbf{Mô tả} & Người dùng giao tiếp với chatbot để được hỗ trợ truy vấn, xem báo cáo, nhận cảnh báo, hoặc được tư vấn chi tiêu \\
	\hline
	\textbf{Điều kiện kích hoạt} & Gửi message trong phần trò chuyện với chatbot \\
	\hline
	\textbf{Pre-condition} & Người dùng đã đăng nhập \\
	\hline
	\textbf{Post-condition} & Người dùng nhận được câu trả lời từ chatbot \\
	\hline
	\textbf{Basic flow} &
	\begin{tabular}{|p{1cm}|p{7.5cm}|}
		\hline
		\textbf{STT} & \textbf{Sự kiện} \\
		\hline
		1 & Người dùng truy cập vào trang chat\\
		\hline
		2 & Người dùng nhập tin nhắn và gửi\\
		\hline
		4 & Chatbot phân tích và xử lý yêu cầu \\
		\hline
		5 & Hệ thống hiển thị phản hồi của chatbot\\
		\hline
	\end{tabular}
	\\
	\hline
	\caption{Use Case: Trò chuyện với chatbot} \label{tab:usecase-chatbot} \\
\end{longtable}


