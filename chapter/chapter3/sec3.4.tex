\section{Biểu đồ hoạt động}
Phần Activity Diagram (Biểu đồ hoạt động) được sử dụng nhằm mô tả chi tiết về luồng hoạt động trong
hệ thống, từ khi người dùng bắt đầu một hoạt động cho đến cách hệ thống phản hồi kết quả cuối cùng. 

Biểu đồ này giúp nhóm phát triển hiểu rõ hơn các bước nghiệp vụ và tương tác giữa người dùng và hệ thống
, đặc biệt trong các ca sử dụng (Use Case) quan trọng. Thông qua biểu đồ này, ta có thể quan sát được toàn bộ 
tiến trình xử lý, bao gồm các hành động mà người dùng thực hiện, các xử lý nội bộ của hệ thống, điều kiện rẽ nhánh, 
cũng như các kết thúc có thể xảy ra. Điều này không chỉ hỗ trợ việc phân tích yêu cầu mà còn giúp lập trình viên 
nắm bắt logic nghiệp vụ để triển khai chính xác hơn.

\textit{Chú thích: Decision trong Activity Diagram theo chuẩn thì được biểu diễn bằng một hình thoi, tuy nhiên do code bằng PlantUml, mặc định decision 
là hình lục giác nên decision trong tài liệu này được để là một hình lục giác. }

\subsection{Đăng ký} 
Activity Diagram "Đăng ký" mô tả quy trình mà người dùng thực hiện để tạo tài khoản 
mới trên hệ thống. Biểu đồ này thể hiện các hoạt động chính như nhập thông tin đăng ký, kiểm tra
tính hợp lệ của dữ liệu, xử lý các trường hợp lỗi và kết thúc bằng việc tạo tài khoản thành công. 

Biểu đồ giúp minh họa rõ các hoạt động giữa người dùng và hệ thống, cho phép nhìn tổng thể các bước,
các quyết định quan trọng cũng như tương tác giữa các đối tượng. 
\begin{figure}[H]
	\centering
	\includegraphics[width=1\textwidth]{figures/register-activity.png}
	\caption{Biểu đồ hoạt động chức năng "Đăng ký"}
\end{figure}

\subsection{Đăng nhập}
Activity Diagram "Đăng nhập" mô tả quy trình mà người dùng thực hiện để đăng nhập 
vào hệ thống. Biểu đồ này thể hiện các hoạt động chính như nhập thông tin đăng nhập, kiểm tra
tính hợp lệ của dữ liệu, xử lý các trường hợp lỗi và kết thúc bằng việc đăng nhập thành công. 

Biểu đồ giúp minh họa rõ các hoạt động giữa người dùng và hệ thống, cho phép nhìn tổng thể các bước,
các quyết định quan trọng cũng như tương tác giữa các đối tượng. 
\begin{figure}[H]
	\centering
	\includegraphics[width=1\textwidth]{figures/login-activity.png}
	\caption{Biểu đồ hoạt động chức năng "Đăng nhập"}
\end{figure}

\subsection{Đăng xuất}
Activity Diagram "Đăng xuất" mô tả quy trình mà người dùng thực hiện để đăng xuất 
khỏi hệ thống. Biểu đồ này thể hiện các hoạt động chính như tương tác của người dùng với hệ thống
tính hợp lệ của dữ liệu, xử lý các trường hợp lỗi và kết thúc bằng việc đăng xuất thành công. 

Biểu đồ giúp minh họa rõ các hoạt động giữa người dùng và hệ thống, cho phép nhìn tổng thể các bước,
các quyết định quan trọng cũng như tương tác giữa các đối tượng. 
\begin{figure}[H]
	\centering
	\includegraphics[width=1\textwidth]{figures/logout-activity.png}
	\caption{Biểu đồ hoạt động chức năng "Đăng xuất"}
\end{figure}

\subsection{Tìm kiếm bài hát}
Activity Diagram "Tìm kiếm bài hát" mô tả quy trình mà người dùng thực hiện để tìm kiếm bài hát muốn nghe 
trên hệ thống. Biểu đồ này thể hiện các hoạt động chính như cách người dùng thao tác với giao diện, nhập thông tin cần tìm kiếm, 
hệ thống xử lý dữ liệu, xử lý các trường hợp lỗi và kết thúc bằng việc tìm thấy một loạt danh sách bài hát thành công. 

Biểu đồ giúp minh họa rõ các hoạt động giữa người dùng và hệ thống, cho phép nhìn tổng thể các bước,
các quyết định quan trọng cũng như tương tác giữa các đối tượng. 
\begin{figure}[H]
	\centering
	\includegraphics[width=1\textwidth]{figures/searchsongs-activity.png}
	\caption{Biểu đồ hoạt động chức năng "Tìm kiếm bài hát"}
\end{figure}

\subsection{Xem bài hát gần đây}
Activity Diagram "Xem bài hát gần đây" mô tả quy trình mà người dùng thực hiện để xem danh sách các bài hát đã nghe 
trên hệ thống. Biểu đồ này thể hiện các hoạt động chính như cách người dùng thao tác với giao diện, hệ thống xử lý dữ liệu, xử lý các trường hợp lỗi 
và kết thúc bằng việc hiện lên một loạt danh sách bài hát thành công. 

Biểu đồ giúp minh họa rõ các hoạt động giữa người dùng và hệ thống, cho phép nhìn tổng thể các bước,
các quyết định quan trọng cũng như tương tác giữa các đối tượng. 
\begin{figure}[H]
	\centering
	\includegraphics[width=1\textwidth]{figures/historysongs-activity.png}
	\caption{Biểu đồ hoạt động chức năng "Xem bài hát gần đây"}
\end{figure}

\subsection{Lưu bài hát vào yêu thích}
Activity Diagram "Lưu bài hát vào yêu thích" mô tả quy trình mà người dùng thực hiện để lưu bài hát vào danh sách các bài hát yêu thích
trên hệ thống. Biểu đồ này thể hiện các hoạt động chính như cách người dùng thao tác với giao diện, hệ thống xử lý dữ liệu 
và kết thúc bằng việc lưu bài hát vào danh sách yêu thích thành công. 

Biểu đồ giúp minh họa rõ các hoạt động giữa người dùng và hệ thống, cho phép nhìn tổng thể các bước,
các quyết định quan trọng cũng như tương tác giữa các đối tượng. 
\begin{figure}[H]
	\centering
	\includegraphics[width=1\textwidth]{figures/addfavorite-activity.png}
	\caption{Biểu đồ hoạt động chức năng "Lưu bài hát vào yêu thích"}
\end{figure}

\subsection{Sửa thông tin cá nhân}
Activity Diagram "Sửa thông tin cá nhân" mô tả quy trình mà người dùng thực hiện để cập nhật thông tin cá nhân như firstname,lastname, mật khẩu, avatar,\dots
. Biểu đồ này thể hiện các hoạt động chính như cách người dùng thao tác với giao diện, hệ thống xử lý dữ liệu 
và kết thúc bằng việc thông tin người dùng được cập nhật thành công. 

Biểu đồ giúp minh họa rõ các hoạt động giữa người dùng và hệ thống, cho phép nhìn tổng thể các bước,
các quyết định quan trọng cũng như tương tác giữa các đối tượng. 
\begin{figure}[H]
	\centering
	\includegraphics[width=1\textwidth]{figures/updateprofile-activity.png}
	\caption{Biểu đồ hoạt động chức năng "Sửa thông tin cá nhân"}
\end{figure}
