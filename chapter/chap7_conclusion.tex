\clearpage
\setcounter{chapter}{6}
\chapter[{TỔNG KẾT VÀ ĐỊNH HƯỚNG TƯƠNG LAI}]{TỔNG KẾT VÀ ĐỊNH HƯỚNG TƯƠNG LAI}
\label{chapter:conclusion}

\section{Tổng kết}	
Ứng dụng âm nhạc InsightTune được tích hợp công nghệ trí tuệ nhân tạo với giao diện
 thiết kế đơn giản, cá nhân hóa trải nghiệm của người dùng.\\
 Hệ thống cung cấp đầy đủ các tính năng cơ bản của một ứng dụng nghe nhạc hiện đại, đồng thời
 tích hợp các công nghệ AI tiên tiến như Chatbot trợ lý ảo và hệ thống gợi ý bài hát, giúp người dùng 
 có một trải nghiệm nghe nhạc tuyệt vời hơn.\\
\subsection{Điểm cộng của hệ thống}
\begin{itemize}
	\item Giao diện thân thiện người dùng phổ thông.
	\item Hệ thống gợi ý bài hát cá nhân hóa dựa trên sở thích người dùng.
	\item Chatbot trợ lý ảo có khả năng giải đáp các thắc mắc, cung cấp thông tin liên quan đến âm nhạc. Người dùng có thể ra lệnh cho Chatbot phát nhạc bằng giọng nói hoặc nhập yêu cầu.
	\item Kiến trúc Microservices giúp hệ thống dễ bảo trì và mở rộng trong tương lai.
	\item Hệ thống được triển khai trên nền tảng đám mây, đảm bảo tính sẵn sàng và khả năng mở rộng.
\end{itemize}
\subsection{Hạn chế của hệ thống}
\begin{itemize}
	\item Chatbot đôi khi trả lời chưa chính xác, hoặc không tự gọi tool.
	\item Hệ thống gợi ý chưa thực sự cá nhân hóa sâu.
	\item Ứng dụng mới chỉ là dạng di động, chưa có phiên bản web hoặc desktop.
	\item Giao diện người dùng còn chưa tối đa hóa trải nghiệm, thiết kế chưa rõ ràng.
	\item Hệ thống chưa có cơ chế caching
\end{itemize}
\section{Định hướng tương lai}
\begin{itemize}
	\item Phát triển thêm phiên bản web và ứng dụng trên desktop.
	\item Hệ thống gợi ý được tối ưu hơn.
	\item Thêm cơ chế caching cho hệ thống.
	\item Giao diện đơn giản, tập trung vào trải nghiệm người dùng.
	\item Phát triển thêm nhiều tính năng như thêm lời bài hát kèm theo bản dịch (nếu cần), ghim ablum,..
	\item Có thể phát triển AI theo hướng mix nhạc tự động dựa vào các bài hát được cung cấp.
\end{itemize}