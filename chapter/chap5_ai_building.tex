\clearpage
\phantomsection

\setcounter{chapter}{4}
\chapter[{XÂY DỰNG HỆ THỐNG AI}]{XÂY DỰNG HỆ THỐNG AI}
\label{chapter:ai}

Chúng tôi xây dựng một hệ thống chatbot agent kết hợp với các tool tìm kiếm và lấy thông tin từ trên 
mạng để cung cấp dữ liệu cho agent với model chính là gemini-2.5-pro. Ngoài ra chúng tôi cũng tự làm thêm tool phát nhạc khi người dùng yêu cầu hoặc có mong muốn nghe nhạc trong app của mình.

Chatbot agent được xây dựng chủ yếu bằng framework mạnh mẽ cho agent là Langgraph và một số các thư viện cần thiết khác để phục vụ cho việc triển khai hệ thống dễ dàng hơn.



Dưới đây là hình ảnh mô tả luồng hoạt đồng cơ bản của hệ thống chatbot agent mà chúng tôi xây dựng:
\begin{figure}
    \centering
    \includegraphics[width=0.4\textwidth]{figures/General_flow.png} 
    \caption{Luồng hoạt đồng của agent trong hệ thống chatbot AI}
    \label{fig:General_flow}
\end{figure}
Khi người dùng nhập input đầu ra, hệ thống sẽ gửi input này tới chatbot thông qua FastAPI. Chatbot sẽ xử lý input này và quyết định xem việc nên dùng tools nào phù hợp.
Khi cần tra cứu thêm thông tin này từ trên mạng, chatbot agent sẽ gọi function calling để gọi đến tool Duckduckgo Search và trả về các link liên quan đến kiến thức mà chatbot cần.
Đến đây, chatbot vẫn chưa có lượng thông tin cần thiết để trả lời bởi vì mới chỉ lấy được các link liên quan. Do đó, chatbot sẽ cần đến tool là Scape Website để lấy dữ liệu từ trang web về ở dạng html.
Lúc này tool Scape Website sẽ trả về input mong muốn cho chatbot. Và rồi chatbot có thông tin và từ đó dựa trên những thông tin ấy trả lời cho người dùng.

Các phần sau đây sẽ mô tả thêm về tool duckduckgo, scrape website.
\input{chapter/chapter5/sec5.1.tex}
\input{chapter/chapter5/sec5.2.tex}
