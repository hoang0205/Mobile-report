\subsection{Client}
\label{subsection:client}
Client là một thành phần chính trong kiến trúc hệ thống. Client nằm thiết bị của người dùng. Nhiệm vụ của Client cung cấp giao diện để người dùng tương tác với hệ thống, ghi nhận dữ liệu đầu và hiển thị dữ liệu của người dùng khi được yêu cầu.

Client trong hệ thống được xây dựng theo kiến trúc \textbf{MVVM (Model - View - ViewModel)} với \textbf{Repository Pattern} nhằm tách biệt các thành phần trong hệ thống, giúp việc phát triển, bảo trì và mở rộng hệ thống trở nên dễ dàng hơn.

Trong kiến trúc này, vai trò của các thành phần được mô tả như sau:
\begin{itemize}
    \item \textbf{Model}: Chứa các lớp đối tượng biểu diễn dữ liệu trong hệ thống. 
    \item \textbf{View}: Hiển thị giao diện và nhận các tương tác từ người dùng. View chỉ tập trung vào hiển thị giao diện và không chứa logic xử lí dữ liệu.
    \item \textbf{ViewModel}: ViewModel nhận yêu cầu, tương tác của người dùng từ View và gửi yêu cầu, nhận dữ liệu từ Repository. ViewModel chịu trách nhiệm xử lí logic, xử lí dữ liệu, quản lý trạng thái và cung cấp dữ liệu cho View. 
    \item \textbf{Repository}: Là lớp trung gian giữa ViewModel và API Server. Repository chịu trách nhiệm thực hiện các thao tác gọi API, đồng thời có thể nhận dữ liệu từ Server dưới dạng JSON và xử lý dữ liệu trước khi trả về cho ViewModel.
\end{itemize}

Khi người dùng tương tác với giao diện-\textbf{View} (ví dụ ấn nút đăng nhập) thì \textbf{View} sẽ gửi yêu cầu đến \textbf{ViewModel}. \textbf{ViewModel} sẽ xử lý yêu cầu này, gọi đến các phương thức trong \textbf{Repository}. \textbf{Repository} sẽ thực hiện lời gọi API đến Server để lấy hoặc nhận dữ liệu. API sẽ trả về dữ liệu cho \textbf{Repository} dưới dạng JSON. \textbf{Repository} nhận, có thể xử lí dữ liệu thành các đối tượng \textbf{Model} và trả về cho \textbf{ViewModel}. \textbf{ViewModel} tiếp tục xử lí dữ liệu nếu cần và cập nhật lại \textbf{View} để hiển thị dữ liệu cho người dùng.
\begin{figure}[H]
    \centering
    \includegraphics[width=0.8\textwidth]{figures/MVVM.png}
    \caption{Kiến trúc MVVM với Repository Pattern trên Client}
    \label{fig:mvvm_repository_pattern}
\end{figure}