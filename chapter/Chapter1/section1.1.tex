\section{Hiện trạng}

Âm nhạc là điều không thể thiếu trong cuộc sống hiện đại ngày nay. Những phần mềm 
nghe nhạc đã ra đời từ rất lâu. Chúng chỉ đơn giản là những trình phát nhạc cơ bản,
cho phép người dùng phát các file âm thanh có sẵn trên thiết bị. Tuy nhiên, với sự
phát triển của công nghệ và Internet, các phần mềm nghe nhạc trực tuyến dần ra đời
thay thế những phần mềm này trong kỷ nguyên của công nghệ đám mây và phi tập trung
hóa.

Những phần mềm nghe nhạc trực tuyến cùng cấu trúc phi tập trung này cho phép người
dùng phát nhạc tốc độ cao từ các hệ thống lưu trữ đám mây mà không cần phải tải các file âm thanh
về thiết bị. Điều này giúp người dùng  tiết kiệm dung lượng lưu trữ trên thiết bị cũng
như dễ dàng tiếp cận với kho âm nhạc khổng lồ từ Internet. Người dùng khi đối mặt với
nhiều lựa chọn khi nghe như vậy lại cảm thấy bối rối vì không biết nên chọn sản phẩm
âm nhạc nào để nghe. Để giải quyết vấn đề này, các phần mềm nghe nhạc ngày nay đang
dần tích hợp các công nghệ trí tuệ nhân tạo (AI) để cá nhân hóa trải nghiệm nghe nhạc
cho người dùng.

Công nghệ trí tuệ nhân tạo (AI) đã và đang được ứng dụng rộng rãi trong nhiều lĩnh
vực khác nhau, bao gồm cả ngành công nghiệp âm nhạc. Trong lĩnh vực âm nhạc, AI có 
hai ứng dụng lớn dựa trên hai nhánh công nghệ của AI: mô hình ngôn ngữ lớn (Large Language Models - LLMs)
và hệ thống gợi ý (Recommender Systems - RS). LLMs mở ra cơ hội tạo ra những Chatbot như
một trợ lý ảo giúp người dùng tìm kiếm, lựa chọn và quản lý âm nhạc một cách dễ dàng.
Điều này tối ưu hóa, đơn giản hóa trải nghiệm người dùng khi tương tác với phần mềm
nghe nhạc. Hệ thống gợi ý (Recommender Systems) sử dụng các thuật toán học máy để phân tích
hành vi nghe nhạc của người dùng, từ đó đề xuất các bài hát, nghệ sĩ hoặc danh sách phát phù hợp với
sở thích cá nhân của họ. Điều này giúp người dùng khám phá âm nhạc mới một cách hiệu quả
và nâng cao trải nghiệm nghe nhạc tổng thể. Những phần mềm nghe nhạc hiện 
nay đều cố gắng tích hợp những công nghệ này để nâng cao và cá nhân hóa
trải nghiệm người dùng. Tuy nhiên, với những phần mềm sẵn có hiện nay, những người 
phát triển độc lập hoặc các nhóm nhỏ sẽ gặp nhiều khó khăn trong việc cải tiến và tích hợp.

Các phần mềm phổ biến hiện nay như Spotify, Apple Music, YouTube Music, Zing MP3, SoundCloud
đều đã có cấu trúc rất lớn và phức tạp. Vấn đề lớn ở đây là chúng thường là phần mềm
có mã nguồn đóng, và việc thêm tính năng hay tích hợp những công nghệ mới vào chúng
là điều khó khăn đặc biệt đối với những nhà phát triển độc lập hoặc các nhóm nhỏ. Do đó,
cộng đồng các nhà phát triển đã cùng nhau tạo ra nhiều dự án mã nguồn mở nhằm mô phỏng lại 
và cung cấp các tính năng tương tự như những phần mềm nghe nhạc lớn hiện nay. Điều này lại
khiến những người mới bắt đầu phát triển khi độ phức tạp của dự án mã nguồn mở này ngày
càng cao và chưa sẵn có một phương thức tích hợp các công nghệ AI tiên tiến vào các dự án này.
Vì vậy, chúng tôi xin giới thiệu \textbf{InsightTune}, một ứng dụng nghe nhạc mã nguồn mở tích hợp AI.

Những đóng góp chính và mục tiêu của dự án của chúng tôi khi phát triển InsightTune bao gồm:
{\bfseries
\begin{itemize}
  \item Phát triển một ứng dụng nghe nhạc mã nguồn mở, miễn phí và dễ dàng tùy chỉnh: 
  front end mobile app với giao diện đơn giản dễ dàng sử dụng, 
  xây dựng back end server theo kiến trúc Microservices như một framework mở rộng sau này.
  \item Tích hợp các công nghệ AI tiên tiến: LLMs(Xây dựng tác tử) và RS (Hệ thống gợi ý sử dụng học tăng cường) trong ứng dụng.
  \item Xây dựng mẫu triển khai trên nền tảng đám mây.
  \item Cung cấp các ca kiểm thử để đảm bảo chất lượng và tính ổn định hệ thống.
\end{itemize}
}