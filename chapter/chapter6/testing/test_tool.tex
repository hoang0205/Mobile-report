\subsection{Công cụ kiểm thử}
Để đảm bảo chất lượng và tính ổn định của hệ thống nghe nhạc trực tuyến InsightTune, yêu cầu đặt ra 
là phải xây dựng bộ kiểm thử, đảm bảo trải nghiệm tốt nhất cho người dùng khi sử dụng hệ thống. 
Kiểm thử hệ thống được thực hiện nhằm đánh giá được toàn bộ chức năng, đảm bảo rằng tất cả các thành 
phần cũng như module được tích hợp trong hệ thống hoạt động đồng bộ, nhất quán, đúng theo yêu cầu đặt ra.
Kiểm thử hệ thống không chỉ là kiểm thử từng chức năng riêng lẻ mà còn chú trọng đến sự tương tác của các thành phần trong hệ thống, dữ liệu 
luân chuyển giữa các module cũng như đáp ứng được các trường hợp có thể xảy ra trong thực tế. 

Để thực hiện kiểm thử hệ thống, nhóm sử dụng một số công cụ nhằm kiểm tra chức năng, hiệu năng, tính ổn định và độ tin cậy 
của ứng dụng: 

\begin{itemize}
    \item Kiểm thử đơn vị (Unit Test) với JUnit
        \begin{itemize}
            \item Mục đích: Đảm bảo logic chính xác của từng thành phần mã nguồn bên phía Backend. 
            \item Áp dụng: Sử dụng Junit để xây dựng các ca kiểm thử tự động cho các logic nghiệp vụ quan trọng như đăng nhập, mã hóa mật khẩu\dots và các thao tác CRUD với CSDl.
        \end{itemize}
    \item Đo lường Độ bao phủ mã (Code Coverage) với JaCoCo:
        \begin{itemize}
            \item Mục đích: Đánh giá chất lượng và mức độ đầy đủ của bộ Unit Test.
            \item Áp dụng: JaCoCo được tích hợp cùng JUnit để tạo báo cáo trực quan, cho biết tỷ lệ phần trăm mã nguồn (lệnh, nhánh, dòng) đã được kiểm thử. Điều này giúp nhóm phát hiện các khu vực mã "chết" hoặc chưa được kiểm thử.
        \end{itemize}
    \item Kiểm thử API với Postman 
        \begin{itemize}
            \item Mục đích: Xác minh rằng các giao tiếp giữa Client (ứng dụng) và Server hoạt động đúng như thiết kế.
            \item Áp dụng: Sử dụng Postman để gửi các HTTP request (GET, POST, PUT, DELETE) đến các API của hệ thống.
        \end{itemize}
    \item Kiểm thử Hiệu năng với JMeter:
        \begin{itemize}
            \item Mục đích: Đánh giá khả năng chịu tải và độ ổn định của hệ thống khi có nhiều người dùng truy cập đồng thời.
            \item Áp dụng: JMeter được dùng để giả lập kịch bản nhiều người dùng cùng lúc thực hiện các hành động như đăng nhập, phát nhạc, và tìm kiếm. Kết quả kiểm thử giúp đo lường thời gian phản hồi của server và tìm ra các "nút thắt cổ chai" (bottleneck) về hiệu năng.
        \end{itemize}
\end{itemize}

Bên cạnh kiểm thử hệ thống, nhóm cũng tiến hành kiểm thử giao diện người dùng để đảm bảo trải nghiệm người dùng mượt mà và tốt nhất.
Để thực hiện kiểm thử giao diện, chúng tôi sử dụng một số công cụ như sau:
\begin{itemize} 
    \item Kiểm thử giao diện người dùng với Jetpack Compose Test:
        \begin{itemize}
            \item Mục đích: Đảm bảo giao diện người dùng với Jetpack Compose sẽ hiển thị chính sác và hoạt động đúng như mong đợi.
            \item Áp dụng: Sử dụng createComposeRule để kiểm thử các giao diện không có trạng thái, các hàm onNodeWithTag và onNodeWithText để tìm các thành phần giao diện,các hàm assert để xác nhận giao diện hiển thị đúng.
        \end{itemize}
    \item Kiểm thử đơn vị (Unit Test) với JUnit4:
        \begin{itemize}
            \item Mục đích: Đảm bảo các hàm, lớp và logic hoạt động chính xác.
            \item Áp dụng: Sử dụng các chú thích @Test, @Before, @Rule để định nghĩa các ca kiểm thử và thực thi các ca kiểm thử.
        \end{itemize}
\end{itemize}