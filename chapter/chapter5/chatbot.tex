\section{Xây dựng trợ lý ảo}

\subsection{Bài toán đặt ra}
Nhận thấy việc phát triển của trí tuệ nhân tạo (AI) và các mô hình 
ngôn ngữ lớn (LLM) đang ngày càng phát triển và mang lại rất nhiều
tiện ích cho người dùng. Và đồng thời khi dùng các ứng dụng âm nhạc hiện
nay thì chúng tôi nhận ra được một vấn đề rằng người dùng thường sẽ 
hay thắc mắc hoặc tò mò về các thông tin liên quan đến bài hát, ca sĩ
 hoặc muốn biết xu hướng âm nhạc hiện nay ra sao. Và việc đang nghe nhạc
 mà lại phải thoát ra để sử dụng một ứng dụng khác để tìm kiếm thông tin 
 khá là bất tiện và làm gián đoạn trải nghiệm người dùng. \\
 Và hiện nay cũng có
 một vài ứng dụng đã nhận thấy được vấn đề này và dần tích hợp chatbot 
 vào ứng dụng phát nhạc nhưng chatbot của họ thường chỉ trả lời dựa trên 
 kiến thức được lưu trữ trong cơ sở dữ liệu có sẵn và không tự cập nhận những
 thông tin mới nhất từ trên mạng. Đây cũng chính là một bất tiện lớn 
 trong việc tối ưu hóa trải nghiệm người dùng bởi người dùng không được 
 cập nhật thông tin một cách chính xác và mới nhất. Do đó chính tôi 
 đã phát triển ra một hệ thống chatbot agent có khả năng tìm kiếm thông 
 tin trên mạng để có thể trả lời người dùng một cách chính xác và 
 bắt kịp thời đại nhanh nhất, không bị giới hạn bởi lượng kiến thức tĩnh.
\subsection{Giải pháp đề xuất}
 Chúng tôi xây dựng một hệ thống chatbot agent kết hợp với các tool tìm kiếm và 
lấy thông tin từ trên mạng để cung cấp dữ liệu cho agent với model chính mà chúng tôi sử dụng là gemini-2.5-pro. 
Điểm mạnh của chatbot của chúng tôi chính là khả năng tìm kiếm thông tin từ trên mạng, có thể tự động cập nhật
những tri thức mới nhất để trả lời cho người dùng mà không bị thiếu kiến thức hay bị sai lệch thông tin
so với các hệ thống chatbot truyền thống dùng các model tĩnh kết hợp RAG (retrieval-augmented generation) để cung
cấp ngữ cảnh cho LLM.
Ngoài ra chúng tôi cũng tự làm thêm tính năng phát nhạc khi người dùng yêu cầu hoặc có mong muốn 
nghe nhạc trong app của mình và có thể yêu cầu bằng giọng nói hoặc nhập yêu cầu bằng văn bản mà không cần phải thao tác thủ công.
\subsection{Công nghệ và luồng hoạt động của chatbot agent}

Chatbot agent được xây dựng chủ yếu bằng framework mạnh mẽ cho agent là Langgraph và một 
số các thư viện cần thiết khác để phục vụ cho việc triển khai hệ thống dễ dàng hơn.
Langgraph giúp chúng tôi xây dựng chatbot agent theo dạng đồ thị (graph) với các node 
đại diện cho các thành phần như tool hay agent. Ngoài ra việc dùng Langgraph giúp chúng tôi 
có thể dễ dàng mở rộng hơn vì nó tối ưu và dễ dàng phát triển để mở rộng thành hệ thống 
multi-agent trong tương lai. 

\subsubsection{Luồng hoạt động chính}
\label{main_workflow}

Trong phần này sẽ mô tả luồng hoạt động cơ bản của hệ thống chatbot agent mà chúng tôi xây dựng.
Sơ đồ luồng hoạt động chính của agent được thể hiện trong hình~\ref{fig:General_flow} dưới đây:

\begin{figure}[H]
    \centering
    \includegraphics[width=0.65\textwidth]{figures/General_flow.png} 
    \caption{Luồng hoạt động của agent trong hệ thống chatbot AI}
    \label{fig:General_flow}
\end{figure}

Khi người dùng nhập input đầu vào, hệ thống sẽ gửi input này tới chatbot thông qua FastAPI. 
Chatbot sẽ xử lý input này và quyết định xem việc nên dùng tools nào phù hợp.

Khi cần tra cứu thêm thông tin này từ trên mạng, chatbot agent sẽ gọi function calling để 
gọi đến tool Duckduckgo Search và trả về các link liên quan đến kiến thức mà chatbot cần.

Đến đây, chatbot vẫn chưa có lượng thông tin cần thiết để trả lời bởi vì mới chỉ lấy được các 
link liên quan. Do đó, chatbot sẽ cần đến tool là Scape Website để lấy dữ liệu từ trang web 
về ở dạng html.

Lúc này tool Scape Website sẽ trả về input mong muốn cho chatbot. 
Và rồi chatbot có thông tin và từ đó dựa trên những thông tin ấy trả lời cho người dùng.
\subsubsection{Quản lí bộ nhớ của Chatbot}
Ở đây chúng tôi quản lí bộ nhớ của chatbot thông qua việc sử dụng LangGraph Checkpointer.
Hình~\ref{fig:memory_flow} mô tả luồng hoạt động của bộ nhớ trong hệ thống chatbot agent của chúng tôi

\begin{figure}[H]
    \centering
    \includegraphics[width=0.5\textwidth]{figures/memory_flow.png} 
    \caption{Luồng hoạt động của bộ nhớ trong hệ thống chatbot AI}
    \label{fig:memory_flow}
\end{figure}

Mỗi cuộc trò chuyện với người dùng sẽ được định danh trong một luồng đã được định danh riêng biệt,
Và Checkpointer sẽ lưu trữ toàn bộ lịch sử tin nhắn của từng luồng đã được định danh. 
Và mỗi khi truy vấn, chatbot sẽ lấy dữ liệu theo luồng đó và cung cấp thông tin cho LLM.
Điều này làm Chatbot duy trì được ngữ cảnh và cá nhân hóa với mỗi lần chat.

\subsubsection{Duckduckgo Tool}

Duckduckgo là một công cụ tìm kiếm, cho phép bạn tra cứu thông tin mà không bị theo dõi hay thu thập dữ liệu cá nhân.
Chatbot sẽ gửi yêu cầu tìm kiếm đến Duckduckgo thông qua chức năng function calling. Duckduckgo sẽ nhận đầu vào là 
một chuỗi kí tự tìm kiếm và rồi sẽ truy vấn để tìm ra 5 link liên quan nhất và trả về cho agent.
Hình~\ref{fig:duckduckgo_flow} mô tả luồng hoạt động của duckduckgo tool trong hệ thống chatbot agent mà chúng tôi phát triển

\begin{figure}[H]
	\centering
	\includegraphics[width=0.6\textwidth]{figures/duckduckgo_flow.png} 
	\caption{Luồng hoạt động của duckduckgo tool trong hệ thống chatbot AI}
	\label{fig:duckduckgo_flow}
\end{figure}

Khi LLM quyết định cần tìm kiếm, gọi Duckduckgo tool kèm theo dữ liệu cần truy vấn. 
Sau đó tool Duckduckgo sẽ trả về tối đa 5 link liên quan nhất đến truy vấn của LLM.
Còn nếu bị lỗi LLM sẽ trả về thông báo lỗi để LLM biết và xử lý, 
có thể gọi lại tool hoặc đưa ra thông báo bị lỗi cho người dùng.
\subsubsection{Scrape Website Tool}

Scrape Website là một tool được xây dựng để lấy dữ liệu từ các trang web dựa trên đường dẫn
 mà LLM cung cấp khi được gọi
thông qua function calling. Tool này giúp chatbot có thể chủ động lấy thông tin từ các 
trang web mà tool Duckduckgo tìm
được để lấy dữ liệu và trả về cho LLM nhằm cung cấp câu trả lời chính xác nhất cho người dùng.
Hình~\ref{fig:scrape_website_flow} mô tả luồng hoạt động của scrape website tool trong hệ thống chatbot agent mà chúng tôi phát triển
\begin{figure}[H]
  \centering
  \includegraphics[width=0.6\textwidth]{figures/scrape_website_flow.png} 
  \caption{Luồng hoạt động của scrape website tool trong hệ thống chatbot AI}
  \label{fig:scrape_website_flow}
\end{figure}

Tool Scrape Website sẽ nhận đầu vào là một đường dẫn từ LLM trả về khi dùng duckduckgo tool để tìm kiếm.
Tool sẽ lấy toàn bộ nội dung website ở dạng văn bản và trả về toàn bộ chúng. 
Lúc này văn bản còn nhiều dữ liệu dư thừa vì ở dạng html, 
nên chúng tôi sẽ sử dụng framework BeautifulSoup để lọc dữ liệu và trả về chuỗi chỉ chứa nội dung của trang website.

\subsubsection{Tính năng phát nhạc}
Để thêm phần trải nghiệm người dùng, chúng tôi đã xây dựng 
thêm tính năng phát nhạc trong ứng dụng chatbot của mình.
Người dùng có thể yêu cầu phát nhạc bằng cách nhập yêu cầu với chatbot
hoặc sử dụng giọng nói để ra lệnh phát nhạc.
Đối với phát nhạc bằng giọng nói, chúng tôi dùng model Whisper của OpenAI để
chuyển đổi giọng nói thành văn bản rồi gửi văn bản này đến Chatbot
để chatbot phân tích và xử lý yêu cầu phát nhạc. Cuối cùng 
trả về cho ứng dụng biết rằng đây là yêu cầu phát nhạc và ứng dụng sẽ 
xử lý phần logic phát nhạc.Chúng tôi phải sử dụng kỹ thuật prompt để
hướng dẫn chatbot cho đầu ra theo đúng định dạng mà ứng dụng 
có thể nhận biết được và phát nhạc đúng yêu cầu của người dùng.


